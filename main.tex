% !TeX spellcheck = en_US
\documentclass[12pt, a4paper]{scrartcl}

%% Language and font encodings
\usepackage[english]{babel}
\usepackage{qtree}
\usepackage[utf8x]{inputenc}
\usepackage[T1]{fontenc}
\usepackage{natbib}
\setcitestyle{authoryear}
\usepackage{stmaryrd}
%\usepackage{times}
\usepackage{subcaption}
\usepackage{graphicx}
\usepackage{fancyvrb}
\newcommand{\cond}[1]{\textsc{#1}}
\usepackage{hyperref}
\usepackage{amssymb}
\usepackage[linguistics]{forest}

\newcommand{\lb}{\llbracket}
\newcommand{\rb}{\rrbracket}
\newcommand{\sx}[1]{$\lb${#1}$\rb$}

%% Sets page size and margins
\usepackage[a4paper, margin=2.5cm]{geometry}

\usepackage{wrapfig,lipsum,booktabs}
\usepackage{times}
%\usepackage[usenames, dvipsnames]{color}
\usepackage{linguex}
\usepackage{cgloss}

% \renewcommand{\refdash}{} %removes dash in references to examples, i.e. (1a) instead of (1-a)

\usepackage[normalem]{ulem}

\begin{document}

\setlength{\Exlabelsep}{0.3em}
\setlength{\SubExleftmargin}{1.3em}
\setlength{\parindent}{0cm} %makes zero indentation at the beginning of paragraphs


\begin{center}
\textbf{Czech binominal \textit{each} and collective set predicates}
\end{center}

\noindent\textbf{Background} In this paper we address the interaction between collective numerals (CN) and de\-ter\-mi\-ner/bi\-no\-mi\-nal \textit{each.} It was noticed in a literature (Dotlačil 2012b) that some types of collectives (collective set predicates following Winter's 2001 terminology) allow limited distributivity effects like the ability to license reciprocal anaphors (e.g. \textit{Bill and Peter, together, carried the piano across each other's lawns}). Such data are not analyzable in the traditional approaches to pluralities and require frameworks interpreting expressions via sets of assignments (Brasoveanu 2008, Nouwen 2003, Dotlačil 2012b a.o.). We follow this trend and describe Czech data (collective interpretation of numerals and their interaction with determiner/binominal \textit{each}) in the PCDRT framework of Dotlačil (2012a,b). Binominal \textit{each} itself poses non-trivial questions for compositional approaches to natural language syntax and semantics and its interaction CN adds another layer of complexity. We argue that the essentially right PCDRT approach has to be enriched with syntactic analysis  to deal with the puzzling Czech data.

\noindent\textbf{Data} Czech numerals have a distinctive subclass of the collective numerals (Dočekal 2012), which \textit{ceteris paribus} enforce collective inferences: compare \Next[a] (ordinary numeral \textit{dva} `two') vs. \Next[b] (collective numeral \textit{dvojice} `twosome'), where the infelicity of the continuation in \Next[b] signals the unavailability of distributive readings with CN; \Next[b] has a collective inference: the two athletes worked together as a team. But even if the collective inference is a part of CN's meaning, they allow for some distributivity  (in contrast to pure collectives); \NNext.\vspace{-4pt}

\ex. \ag. \textbf{Dva} sportovci vyhráli 2 medaile, $\checkmark$\hspace{-3pt} první zlato a stříbro, druhý stříbro a bronz.\\
two athletes won 2 medals {} first gold \& silver second silver \& bronze\\ \\
`Two athletes won 2 medals, the first one G \& S, the second one S \& B.'
\b. \textbf{Dvojice} sportovců vyhrála 2 medaile, {\#}\hspace{-2pt} první zlato a stříbro, druhý stříbro\ldots
% Two-CN athletes won 3 medals, \#first gold and silver, second silver\ldots.

\exg. \textbf{Dvojice} /\#\hspace{-2pt} \textbf{Skupina} podezřelých zradila jeden druhého.\\
twosome {} group suspects.\textsc{gen} betrayed one other.\\
(Intended:) `The people within the twosome / group of suspects betrayed one another.'\vspace{-4pt}

The puzzling pattern we aim to address is presented in \Next: \Next[a] has the expected collective reading but the determiner \textit{each} in \Next[b] allows distributive reading even with CN. But as \Next[c] shows such a distributive reading is unavailable with binominal \textit{each}. The grammaticality of \Next[b] is to some extent expected after \Last but then unacceptability of \Next[c] is surprising. Compare the perfectly grammatical \Next[d] with cardinal numeral substituting the CN.\vspace{-4pt}

\ex. \ag. \textbf{Dvojice} sportovců vyhrála 3 medaile.\label{ex:dvojice-cum}\\
twosome athletes.\textsc{gen} won.\textsc{sg.fem} 3 medals.\\\hfill \textbf{*distributive}\\\vspace{-4pt}
\bg. \textbf{Každý} \textbf{z} \textbf{dvojice} sportovců vyhrál 3 medaile.\label{ex:dvojice-det-each}\\
each of twosome.\textsc{gen} athletes.\textsc{gen} won.\textsc{sg.masc} 3 medals\\\hfill $\checkmark$\textbf{distributive}\\\vspace{-4pt}
\bg. *\textbf{Dvojice} sportovců vyhrál(a) \textbf{každý/á} 3 medaile.\label{ex:dvojice-bin-each}\\
twosome athletes.\textsc{gen} won.\textsc{sg.masc(fem)} each.\textsc{sg.masc/fem} 3 medals\\ \\\vspace{-4pt}
\bg. \textbf{Dva} sportovci vyhráli \textbf{každý} 3 medaile.\label{ex:dva-bin-each}\\
two athletes won.\textsc{pl.masc} each.\textsc{sg.masc} 3 medals\\\hfill$\checkmark$\textbf{distributive}\\

\vspace{-8pt}\textbf{Analysis} The core assumptions of our analysis are (i) Dotlačil's PCDRT and, for the case of \Last[d], (ii) the structure shown in the figure below, involving the deletion of a definite NP anaphoric to the key (under partial matching with the key, modulo number); while the key controls agreement on the verb (and case-marking on `each'), the deleted NP controls the number on `each'; and (iii) the lexical entries for determiner and binominal \textit{každý} `each' listed in \ref{def:each}.\vspace{-4pt}

\ex. \label{def:each}\a. \sx{\textsc{det}-každý$^{u_n}$}${}=\lambda P_{rt}\lambda Q_{rt}.\delta_{u_n}(P(u_n)) \wedge Q(u_n)$\label{def:each-det}
\b. \sx{\textsc{binom}-každý$^{u_m}$}${}=\lambda v_r\lambda P_{rt}\lambda Q_{rt}.[u_m\mid] \wedge \delta_{v}(P(u_m)) \wedge Q(u_m)$\label{def:each-bin}

\textit{Analysis of \ref{ex:dvojice-cum}:} The subject `twosome of athletes' ($\lambda Q_{rt}.[u_1| \#(u_1)=2 \wedge \textsc{athletes}\{u_1\}] \wedge Q(\bigcup u_1)$) selects the VP ($\lambda v_r[u_2 | \#(u_2)=2 \wedge \textsc{medals}\{u_2\} \wedge \textsc{win}\{v,u_2\}]$), which results in \Next. The only addition (to standard numerical conditions of PCDRT) is the collective inference stemming from the quantifier denotation of the CN, where the collective set satisfaction is required in the nuclear scope -- the external argument in this case ($\textsc{win}\{\bigcup u_1,u_2\}$).\vspace{-4pt}

\ex. $[u_1, u_2| \#(u_1)=2\wedge\textsc{athletes}\{u_1\} \wedge \#(u_2)=3 \wedge \textsc{medals}\{u_2\} \wedge \textsc{win}\{\bigcup u_1,u_2\}]$\vspace{-4pt}


 
\textit{Analysis of \ref{ex:dvojice-det-each}:} We propose that the preposition \textit{z} `from/of' turns predicates of groups to predicates of their parts -- $\lambda P_{rt}\lambda v_r.[| v\subseteq P]$, thereby creating a property that can be selected by `each'. The preposition operates on the predicative meaning of the CN (we follow the consensus in approaches to pluralities, where collectivity/distributivity always targets the predicates), with the collective inference targeting the CN itself ($\lambda w_r[|\#(w)=2 \wedge \textsc{athletes}\{\bigcup w\}]$). When the VP (as above) is selected by the quantificational subject ($\lambda Q_{rt}.[v|\delta_{v}([|\lambda v_r.[v \subseteq \lambda w_r[|\#(w)=2 \wedge \textsc{athletes}\{\bigcup w\}]]]) \wedge Q(v)$), we get \Next.%: the condition \textsc{athlete}\{v\} results partitioning, the atomicity of the condition reflects the morfosyntactic singular on \textit{každý} `each'.\vspace{-4pt}

\ex. $[v,u_2|  \textsc{athlete}\{v\} \wedge \delta_{v}([|\lambda v_r.[v \subseteq \lambda w_r[|\#(w)=2 \wedge \textsc{athletes}\{\bigcup w\}]]]) \wedge \#(u_2)=3 \wedge \textsc{medals}\{u_2\} \wedge \textsc{win}\{v, u_2\}])]$\vspace{-4pt}
\z.

 \begin{wrapfigure}{r}{0.35\textwidth}
 \vspace{-20pt}
 \begin{footnotesize}
% \Tree [.S [.DP$_1$ key ] [.VP$_1$ [.V won ] [.DP$_2$ [.{} [.Det each ]  [.\sout{NP$_3$} [.the u$_1$ ] [.athlete ] ] ] [.NP$_2$ share  ]  ] ] ]
\begin{forest}
[S [DP$_1$ [\textsf{key}.\textsc{pl}] ] [VP$_1$ [V [won.\textsc{pl}] ] [DP$_2$ [{} [Det [each.\textsc{sg}] ]  [NP$_3$\\$u_1$ [\sout{the athlete.\textsc{sg}}, roof] ] ] [NP$_2$ [\textsf{share}]]  ]  ] ]
\end{forest}
\end{footnotesize}
\vspace{-15pt}
\end{wrapfigure}
 
 \textit{Analysis of \ref{ex:dva-bin-each}:} We argue that the syntax of Czech binominal \textit{každý} `each' is essentially the same as proposed by Dotlačil (2012a). That \textit{každý} + the share form a constituent (as opposed to the floating Q \textit{všichni} `all' + direct object) is demonstrated in \ref{each-all}, where they have been fronted as a single unit. The difference to Dotlačil's analysis (to English \textit{each}) is that the the anaphoricity of the Czech \textit{každý} is represented in the syntax -- by an NP that is anaphoric to the key and which is deleted under partial (modulo number) identity with the key. This NP (whose exact semantics will be provided in the talk) licenses the singular morphology on \textit{každý.} The resulting meaning of the quantifier DP$_2$ is $\lambda Q_{rt}[u_2 | \wedge \delta_{u_1}([u_2| \#(u_2)=3 \wedge \textsc{medals}\{u_2\}] \wedge Q(u_2)$ and the meaning of \ref{ex:dva-bin-each} as a whole is given in \ref{mean:dva-bin-each}.\vspace{-4pt}
 
 \exg.  [\hspace{-2pt} Každý /*\hspace{-2pt} Všichni 3 medaile] vyhráli jen čeští sportovci.\\
 {} each.\textsc{sg.masc} {} all.\textsc{pl.masc} 3 medals won.\textsc{pl} only Czech athletes\\ \\
 (Intended:) `Only the Czech athletes have (all) won (each) three medals.'\label{each-all}
 
 \ex. $[u_1,u_2|\#(u_1)=2 \wedge \textsc{athletes}\{u_1\} \wedge \delta_{u_1}([\#(u_2)=3 \wedge \textsc{medals}\{u_2\}])] \wedge \textsc{win}\{u_1,u_2\}$\label{mean:dva-bin-each}\vspace{-4pt}
 
 
 
 \textit{Analysis of \ref{ex:dvojice-bin-each}:} The reason behind the ungrammaticality of this example is that the subject and its scope impose conflicting requirements \textit{qua} collectivity/distributivity: while the subject requires collectivity in its nuclear scope -- \Next[a], binominal \textit{každý} (VP$_1$ node in \Next[b]) dictates quantification over key's atoms.\vspace{-4pt}
%-----------------
%\begin{wrapfigure}{l}{0.37\linewidth}
% \newline

% \begin{minipage}{.45\textwidth}

% tree

% \end{minipage}
% \begin{minipage}{.55\textwidth}







\ex. \a. \sx{DP$_1$ of \ref{ex:dvojice-bin-each}}$=\lambda Q_{rt}.[u_1| \#(u_1)=2 \wedge \textsc{athletes}\{u_1\}] \wedge Q(\bigcup u_1)$
\b. \sx{VP$_1$ of \ref{ex:dvojice-bin-each}}$=\lambda v_r[u_2 | \delta_{u_1}([\#(u_2)=3 \wedge \textsc{medals}\{u_2\}]) \wedge \textsc{win}\{v,u_2\}]$


% \end{minipage}
\vspace{-3pt}


%\end{wrapfigure}

\scriptsize
\noindent\textbf{Selected references:} Brasoveanu, A. (2008). Donkey pluralities. \textit{L\&P}, 31(2), 129-209 $\bullet$ Dočekal, M. (2012). Atoms, groups and kinds in Czech. ALH, 59(1-2), 109-126 $\bullet$ Dotlačil, J. (2011). Fastidious distributivity. In \textit{SaLT} (Vol. 21, pp. 313-332) $\bullet$ Dotlačil, J. (2012a). Binominal each as an anaphoric determiner. In \textit{SuB} (Vol. 16, pp. 211-224) $\bullet$ Dotlačil, J. (2012b). Reciprocals distribute over information states. \textit{JoS}, 30(4), 423-477 $\bullet$ Nouwen, R. W. F. (2003). Plural pronominal anaphora in context (PhD thesis) $\bullet$ Winter, Y. (2001). Flexibility principles in Boolean semantics. Cambridge, MA: MIT press.

% \newpage
% \normalsize

% \textbf{Analysis:} basic ingredients of PCDRT we need are the following: expressions are interpreted relative to sets of assignments (plural information states). In this framework binominal and determiner \textit{each} introduce a distributive operator $\delta$ in \Next[a] (after DOTLAČIL) which filters-out all assignments with non-atomic denotation (the part $\mid \bigcup u_nI\mid = 1$). Binominal \textit{each} in \Next[b] (after DOTLAČIL) distributes over atoms in key's denotation (it is anaphoric to key) and it requires that the share denotation has the right cardinality for each atom in the key.

% \begin{wraptable}{l}{0.37\linewidth}
% \begin{tabular}{|c|c|c|}
% \hline 
% Info state J & u$_1$ & u$_2$ \\ 
% \hline 
% j$_1$ & athlete$_1$ & medal$_1$ \\ 
% \hline 
% j$_2$ & athlete$_1$ & medal$_2$ \\ 
% \hline 
% j$_3$ & athlete$_1$ & medal$_3$ \\ 
% \hline 
% j$_4$ & athlete$_2$ & medal$_4$ \\ 
% \hline 
% j$_5$ & athlete$_2$ & medal$_5$ \\ 
% \hline 
% j$_6$ & athlete$_2$ & medal$_6$ \\ 
% \hline 
% \end{tabular} 
% \caption{Figure 1: distributive info states for \ref{ex:dvojice-det-each}}
% \end{wraptable}
% %------------------------------------------
% The determiner \textit{each} in \Next[c] doesn't introduce a new discourse referent (dref) like the binominal \textit{each} but it requires all the drefs satisfying restrictor to satisfy its scope one by one. In other words: beside anaphoricity there is no difference between binominal and distributive \textit{each} (expected as they are both one lexical item). As for CN, we formalize their collective inference as the imposement on the predicate of the sentence to be satisfied in all assignments of a dref u$_n$ ($\bigcup u_n$). The formalization of \ref{ex:dvojice-cum} is in \NNext[a]: it is a cumulative reading which impose (collective set) satisfaction of the predicate \textit{WIN} (on its first argument) to all assignments of u$_1$. The numerical cardinality is the same as in standard PCDRT condition on numerals, the only distinction is the collective inference formalized as $WIN\{\bigcup u_1,u_2\}$. Formalization of \ref{ex:dvojice-det-each} is in \NNext[b]: the distributive interpretation is grammatical as the CN isn't argument of the verb but is a restrictor of the determiner \textit{each}, so there is no problem with the collective entailment imposed on the verb -- one verifying info state for \ref{ex:dvojice-det-each} is in the Figure 1. Lastly \NNext[c] formalizes the sentence \ref{ex:dvojice-bin-each} and explains its ungrammaticality: the crucial problem here is that the CN is part of the argument NP which is the subject of the verb, so it enforces the collective set interpretation in the subject verbal argument ($\bigcup u_1$) which of course clashes with the semantics of $\delta$ requiring each atom of the key to satisfy the scope. 
% %------------------------------------------

% \ex. \a. $\delta_{u_n}(D):=\lambda I_{st}\lambda J_{st}.u_nI=u_nJ \wedge \forall d \in u_nI(\mid \bigcup u_nI\mid = 1 \wedge D(I\mid u_{n=d})(J\mid u_{n=d}))$
% \b. $\lb each^{u_m}_{u_n}\rb=\lambda P_{rt}\lambda Q_{rt}.[u_m\mid] \wedge \delta_{u_n}(P(u_m)) \wedge Q(u_m)$
% \c. $\lb each^{u_n}\rb=\lambda P_{rt}\lambda Q_{rt}.\delta_{u_n}(P(u_m)) \wedge Q(u_m)$

% \ex. \a. $[u_1, u_2\mid 2 ATOMS\{\bigcup u_1\} \wedge ATHLETE(s)\{u_1\} \wedge 3 ATOMS\{\bigcup u_2\} \wedge MEDAL(s)\{u_2\} \wedge WIN\{\bigcup u_1,u_2\}]$
% \b. $([\mid ATHLETE(s)\{u_1\} \wedge 2 ATOMS\{\bigcup u_1\}] \wedge \delta_{u_1} ([\mid ATOM \{\bigcup u_1\}] \wedge [\mid 3 ATOMS\{\bigcup u_2\} \wedge MEDAL(s)\{u_2\}] \wedge [\mid WIN\{u_1,u_2\}]$
% \c. $([\mid ATHLETE(s)\{u_1\} \wedge 2 ATOMS\{\bigcup u_1\}] \wedge \delta_{u_1} ([\mid ATOM \{\bigcup u_1\}] \wedge [\mid 3 ATOMS\{\bigcup u_2\} \wedge MEDAL(s)\{u_2\}] \wedge [\mid WIN\{\bigcup u_1,u_2\}]$

% \textbf{Syntax:} anaphoricity in \ref{ex:dva-bin-each}: agreement in case and gender but sg instead of pl number of its anaphoric key.


% % Radek:

% \ex. Ty slepice snesly každá tři vejce.\\
% \Tree [.VP \qroof{ty slepice}.NP$_1$ [.{} $i$ [.VP snesly [.NP$_2$ [.NP$_3$ [.Det každá ] [.NP$_4$ [.Dem ta $i$ ] \qroof{slepice}.NP$_5$ ] ] \qroof{tři vejce}.NP$_6$ ] ] ] ] \medskip

% We assume that demonstratives subcategorize for a silent index (Elbourne 2008, Schwarz 2009, a.o.) which can be bound by a c-commanding expression (here: the subject NP \textit{ty slepice}). The reference of this index is then equated with the denotation of the NP-argument of the demonstrative. We assume here for simplicity that the final value of the demonstrative is simply a referent and that the demonstrative contributes no uniqueness (it merely contributes anaphoricity, no uniqueness/maximality; cf. Šimík 2016 for an argument from Czech demonstratives). The NP simply contributes a free variable, restricted by its descriptive content (Heim 1982). The relation $<$ (part of) is contextually supplied (and could be reformulated to become compatible with Jakub's account; if I'm not mistaken, he doesn't assume a mereological analysis of sg vs. pl, or does he?). For independent motivation for contextual determination (and the associated flexibility) of the relation between the NP-denotation and its antecedent in demonstrative NPs, see Šimík 2016.

% \ex. \a. \sx{ta}${}=\lambda x\lambda v[v : v < x]$\hfill $<$ is a contextually supplied relation\medskip
% \b. \sx{Dem}$^g=\lambda v[v : v < g(i)]$\hfill $g(i) ={}$\sx{NP$_1$}\medskip
% \b. \sx{NP$_5$}${}= u_m : \textsc{hen}(u_m)$\medskip
% \b. \sx{NP$_4$}$^g = u_m : \textsc{hen}(u_m)\wedge u_m < g(i)$\medskip
% \b. \sx{Det}${}=\lambda v\lambda P\lambda Q[\delta (P(v))\wedge 	Q(v)]$

% \Tree[.VP [.rt,t [.EC ] [.{dva sportovci} ]] [.rt [.každý ] [.rt [.vyhráli ] [.rt,t [.EC ] [.{3 medaile} ] ] ]] ]

% \newpage

% \ex. Dvojice sportovců vyhrála 3 medaile.

% \Tree[.S [.{} \qroof{EC$^{u_1}$ two-ice sportovců}.DP$_1$ ] [.VP$_1$ [.V won ] [.{} \qroof{EC$^{u_2}$ 3 medals }.DP$_2$ ] ] ]

% \ex. \a. \sx{S}$=[u_1, u_2| \#(u_1)=2\wedge\textsc{athletes}\{u_1\} \wedge \#(u_2)=3 \wedge \textsc{medals}\{u_2\} \wedge{}$\\$\textsc{win}\{\bigcup u_1,u_2\}]$
% \b. \sx{DP$_1$}$=\lambda Q_{rt}.[u_1| \#(u_1)=2 \wedge \textsc{athletes}\{u_1\}] \wedge Q(\bigcup u_1)$
% \c. \sx{VP$_1$}$=\lambda v_r[u_2 | \#(u_2)=2 \wedge \textsc{medals}\{u_2\} \wedge \textsc{win}\{v,u_2\}]$
% \d. \sx{DP$_2$}$=\lambda Q_{rt}.[u_2| \#(u_2)=3 \wedge \textsc{medals}\{u_2\}] \wedge Q(u_2)$

% \ex. Každý z dvojice atletů vyhrál 3 medaile.

% \Tree[.S [.DP$_1$ [.každý ] [.PP [.P z ] [.NP$_1$ ] ] ] [.VP$_1$ ] ]

% \ex. \a. \sx{NP$_1$}$=\lambda w_r[|\#(w)=2 \wedge ATHLETE(s) \{w\}]$
% % * <radek.simik@hu-berlin.de> 2018-05-07T09:52:06.166Z:
% % 
% % > $=\lambda w_r[|\#(w)=2 \wedge ATHLETE(s) \{w\}]$
% % Tato semantika je tedy identicka jako kdyby to byli "dva atleti"?
% % 
% % ^.
% \b. \sx{P}$=\lambda P_{rt}\lambda v_r.[| v \subseteq P(\bigcup v)]$
% % * <radek.simik@hu-berlin.de> 2018-05-07T09:48:51.691Z:
% % 
% % > $=\lambda P_{rt}\lambda v_r.[| v \subseteq P(\bigcup v)]$
% % Obavam se ze tady to nevychazi typove: Kdyz uz, tak bysme asi museli psat {v}, protoze v neni set (nebo se mylim?). Druhy, vetsi problem je, ze P(Uv) je klasickeho typu t a tim padem v (popr. {v}) nemuze byt jeho subset...
% % 
% % ^.
% \c. \sx{PP}$=\lambda v_r.[| v \subseteq \lambda w_r[|\#(w)=2 \wedge ATHLETE(s) \{w\})(\bigcup v)]$
% \d. \sx{každý$^{u_n}$}$=\lambda P_{rt}\lambda Q_{rt}.[u_n\mid] \wedge \delta_{u_n}(P(u_m)) \wedge Q(u_m)$
% % * <radek.simik@hu-berlin.de> 2018-05-07T10:13:49.683Z:
% % 
% % > \sx{každý$^{u_n}$}$=\lambda P_{rt}\lambda Q_{rt}.[u_n\mid] \wedge \delta_{u_n}(P(u_m)) \wedge Q(u_m)$
% % Tady u toho kazdy mas nevazanou promennou u_m. Nejsem si ted jaksi jisty, co to ma znamenat, resp. jak to opravit.
% % 
% % ^.
% \e. \sx{DP$_1$}$=\lambda Q_{rt}.[u_1\mid \wedge \delta_{u_1}(\lambda v_r.[| v \subseteq \lambda w_r[|\#(w)=2 \wedge ATHLETE(s) \{w\})(\bigcup v)](u_1))] \wedge Q(u_m)$
% \newpage

% % * <radek.simik@hu-berlin.de> 2018-05-07T10:05:26.683Z:
% % 
% % Nize pokus o korekci tveho (11b/c) - (a). Fungovat by to melo v pripade, ze zachovame [[dvojice studentu]] = [[dva studenti]], jak to ted podle me mas. Pokud [[dvojice studentu]] bude mit v sobe podminku, ze ma v extenzi jenom skupiny, pak by nam toto P nepomohlo. Museli bychom definovat operator, ktery ze skupiny vytahne jeji cleny (rekneme UNGROUP). Pak by bylo jako v (b) nize.
% % 
% % ^.

% Tady navrh na korekci vyznamu P, viz koment vyse.

% \ex.[(I)] \a. \sx{P}${}=\lambda P_{rt}\lambda v_r.[| v\subseteq P]$
% \b. \sx{P}${}=\lambda P_{rt}\lambda v_r.[| \textsc{ungroup}(P)(v)]$

% \ex. Dva atleti vyhráli každý 3 medaile.

% \Tree[.S [.DP$_1$ ] [.VP [.V won ] [.DP$_2$ [.{} [.Det každý:$\delta$ ]  [.NP$_3$ [.ten u$_1$ ] [.atlet ] ] ] [.NP$_2$  ]  ] ] ]

% \ex. \a. \sx{každý$^{u_m}_{u_n}$}=$\lambda P_{rt}\lambda Q_{rt}.[u_m\mid] \wedge \delta_{u_n}(P(u_m)) \wedge Q(u_m)$
% \b. \sx{DP$_2$}$=\lambda Q_{rt}[u_2 | \wedge \delta_{u_1}([u_2| \#(u_2)=3 \wedge MEDAL(s)\{u_2\}] \wedge Q(u_2)$

% Tady pokus o formulaci toho kompletnejsiho ceskeho each a semantiky \textit{ten atlet} (ackoliv tam nevim, jestli to uvozuje jen toho referenta,  nebo cely informacni stav). Vyznam atlet je proste pridany jako podminka na hodnotu toho referenta (in line with Heim 1982). Da se to prip. udelat i cele kompozicne (jak jsem uz nacrtnul vyse, v (7)).

% \ex.[(II)] \a. \sx{každý$^{u_m}$}=$\lambda v_r\lambda P_{rt}\lambda Q_{rt}.[u_m\mid] \wedge \delta_{v}(P(u_m)) \wedge Q(u_m)$
% \b. \sx{ten atlet}${}=u_1$ (such that $\textsc{athletes}\{u_1\}$)


% \ex. \#Dvojice atletů vyhrála každý 3 medaile.

% \Tree[.S [.DP$_1$ ] [.VP [.V won ] [.DP2 [.{} [.Det každý:$\delta$ ]  [.NP$_3$ [.ten u$_1$ ] [.atlet ] ] ] ] ] ]

% \ex. \a.\sx{DP$_1$}$=\lambda Q_{rt}.[u_1| \#(u_1)=2 \wedge ATHLETE(s)\{u_1\}] \wedge Q(\bigcup u_1)$
% \b. \sx{VP$_1$}$=\lambda v_r[u_2 | \wedge \delta_{u_1}([u_2| \#(u_2)=3 \wedge MEDAL(s)\{u_2\}]$

\end{document}

